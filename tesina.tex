\documentclass[]{article}
\usepackage{lipsum}  


% Title Page
\title{PASTEL}
\author{{Augello Andrea} \and {Bafumo Francesco} \and{La Martina Marco}}


\begin{document}
\maketitle

\section{Introduzione}

{[...]} linguaggio per gestire le interazioni tra dispositivi IoT [...] sistemi ciberfisici complessi che necessitano una conoscenza dello stato globale del sistema per coordinarsi. Infatti is sitema potrebbe includere dispositivi con risorse limitate e scarse capacità computazionali, che quindi non sono in grado di memorizzare lo stato del resto dei sensori ed effettuare decisioni complesse.

\nocite{gaglio2017dc4cd}

\section{Stato dell'arte}
\lipsum[2]
\nocite{libes1991expect, libes1990expect}

\section{Descrizione del progetto}
\lipsum[3]
\subsection{Analisi dei requisiti}
\subsection{Scelte progettuali}
\section{Caratteristiche del linguaggio}
\subsection{Grammatica}
\subsection{Descrizione del parser}
\subsection{Casi d'uso}
\subsection{Risultati ottenuti}
\section{Conclusioni}




\bibliographystyle{unsrt}
\bibliography{references}

\end{document}          
