% !TeX spellcheck = it_IT
\documentclass[]{article}
\usepackage{color, soul}
\newcommand{\hlc}[2][yellow]{ {\sethlcolor{#1} \hl{#2}} }
\usepackage[T1]{fontenc}
\usepackage[utf8]{inputenc}



% Title Page
\title{PASTEL}
\author{{Augello Andrea} \and {Bafumo Francesco} \and{La Martina Marco}}


\begin{document}
\maketitle

\section{Introduzione}

{[...]} linguaggio per gestire e coordinare le interazioni tra dispositivi IoT [...] sistemi ciberfisici complessi che necessitano una conoscenza dello stato globale del sistema per coordinarsi. Infatti il sitema potrebbe includere dispositivi con risorse limitate e scarse capacità computazionali, che quindi non sono in grado di memorizzare lo stato del resto dei sensori ed effettuare decisioni complesse. 

Un altro contesto in cui PASTEL può essere utile è testare in modo replicabile il corretto comportamento interattivo di un sistema ciberfisico: il linguaggio proposto infatti può facilmente simulare l'output di molti sensori ed inviarlo agli attuatori.


\section{Stato dell'arte}

I dispositivi IoT comunemente eseguono protocolli semplici come CoAP, REST e MQTT~\cite{tandale2017empirical}. \hlc[cyan]{[Inserire descrizione dei protocolli, magari qualche altra citazione]}

Il linguaggio proposto può facilmente essere utilizzato con i protocolli precedenti, ma l'ambiente target è quello dei dispositivi in grado di eseguire codice simbolico~\cite{gaglio2017dc4cd}.  

[...]L'equivalente di \texttt{expect}~\cite{libes1991expect, libes1990expect} per WSN

\section{Descrizione del progetto}
\subsection{Analisi dei requisiti}
\subsection{Scelte progettuali}
\section{Caratteristiche del linguaggio}
\subsection{Grammatica}
\subsection{Descrizione del parser}
\subsection{Casi d'uso}
\subsection{Risultati ottenuti}
\section{Conclusioni}




\bibliographystyle{unsrt}
\bibliography{references}

\end{document}          
